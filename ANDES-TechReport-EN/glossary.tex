\usepackage[nonumberlist]{glossaries}

\makenoidxglossaries

\newglossaryentry{Catch Card} {
  name=Catch Card,
  description={Not sure yet, I think it's related to species and how they are measured/detailed}
  }

\newglossaryentry{Cruise} {
  name=Cruise,
  description={Same as \gls{Mission}},
  }

\newglossaryentry{Current Set} {
  name=Current Set,
  description={The \gls{Set} that is currently occuring in time. It has begun and is not finished. There can only be one \gls{Current Set} at a time.},
  see={Set}
  }

\newglossaryentry{Closed Set} {
  name=Closed Set,
  description={An \gls{Open Set} Can be closed if the data it contains (eg, from \gls{Catch Card}) does not trigger a \gls{Set Flag}.},
  see={Set}
  }


\newglossaryentry{Mission} {
  name=Mission,
  description={Same as \gls{Cruise}}
  }

\newglossaryentry{Open Set} {
  name=Open Set,
  description={A \gls{Set} that has not been closed. All new Sets will start in an open state and will remain so untill closed. Sets that are activated will automatically be opened. The \gls{Current Set} is always open.},
  see={Set}
  }

\newglossaryentry{Set} {
  name=Set,
  description={A \gls{Set} contains fishing activity}
  }
  
\newglossaryentry{Set Flag} {
  name=Set Flag,
  description={A flag used to indicate that the data contained within the set has failed to pass a specific validation tests. CLosing an \gls{Open Set} with active flags can be done by overriding the }
  }
  
\glsaddall
\glsaddallunused
