\usepackage[nonumberlist,acronym,numberedsection]{glossaries}

\makenoidxglossaries

\newglossaryentry{Andes lead} {
  name=Andes lead,
  description={A person with a functional understanding of the inner workings of Andes.
   This person has the knowledge and user-rights that permit them to configure a \gls{Mission}, define a \gls{Protocol}, \gls{Sampling Req}, etc.}
}

%%% The glossary entry the acronym links to
\newglossaryentry{apig}{name={API},
    description={An Application Programming Interface (API) is a particular set of rules and specifications that a software program can follow to access and make use of the services and resources provided by another particular software program that implements that API}}

%%% define the acronym and use the see= option
\newglossaryentry{api}{type=\acronymtype, name={API}, description={Application
Programming Interface}, first={Application Programming Interface (API)\glsadd{apig}}, see=[Glossary:]{apig}}


%%% The glossary entry the acronym links to
\newglossaryentry{azmpg}{name={AZMP},
    description={The Atlantic Zone Monitoring Program collects and analyses the biological, chemical and physical oceanographic field data in the four Altantic regions of Fisheries and Oceans Canada}}

%%% define the acronym and use the see= option
\newglossaryentry{azmp}{type=\acronymtype, name={AZMP}, description={Atlantic Zone Monitoring Program}, first={Atlantic Zone Monitoring Program (AZMP)\glsadd{azmpg}}, see=[Glossary:]{azmpg}}


\newglossaryentry{Bridge} {
  name=Bridge,
  description={
    The Bridge app is meant to be used by the navigation officer while the fishing officer deploys and retrieves the trawl. The operator inputs fishing events (eg., net deployed, doors deployed, winches locked, net on bottom, haul back, net off bottom, doors recovered, net recovered) or actions (eg., net on/off bottom) directly to Andes via the Bridge app.
  }
}

\newglossaryentry{Catch} {
  name=Catch,
  description={
    A table of the Ecosystem Surveys app. A Catch table contains a link to \gls{Set} and \gls{Species} entries, as well as an optional: speciment count, unweighted baskets, relative abundance category and an invertabrate catch ratio. Catches can be subdivided into sub-catches having a parent-children relationship.
  }
}

\newglossaryentry{Catch Card} {
  name=Catch Card,
  description={
    Catch details recorded prior to detailed sampling.
  }
}

\newglossaryentry{CCG} {
  % type=\acronymtype,
  name=CCG,
  description={The Canadian Coast Guard.},
  text={CCG},
  first={Canadian Coast Guard (CCG)}
}

\newglossaryentry{Cruise} {
  name=Cruise,
  description={Same as \gls{Mission}},
}

\newglossaryentry{Current Set} {
  name=Current Set,
  description={
    The \gls{Set} that is currently occuring in time.
    It has begun and is not finished.
    There can only be one \gls{Current Set} at a time.
  },
  see={Set}
}

\newglossaryentry{Closed Set} {
  name=Closed Set,
  description={
    An \gls{Open Set} Can be closed if the data it contains (eg, from \gls{Catch Card}) does not trigger a \gls{Set Flag}.
  },
  see={Set}
}

 \newglossaryentry{Configuration Preset} {
  name=Configuration Preset,
  description={
    Configuration for a particular deployment scenario. Configuration presets include details on the backup locations, and peripherals like GPS, sonars, label printers, etc.
  }
}


%%% The glossary entry the acronym links to
\newglossaryentry{dfog}{name={DFO},
    description={Fisheries and Oceans Canada is a department of the Government of Canada that is responsible for developing and implementing policies and programs in support of Canada's economic, ecological and scientific interests in oceans and inland waters.}}

%%% define the acronym and use the see= option
\newglossaryentry{dfo}{type=\acronymtype, name={DFO}, description={Fisheries and Oceans Canada, formerly the Department of Fisheries and Oceans}, first={Fisheries and Oceans Canada (DFO)\glsadd{dfog}}, see=[Glossary:]{dfog}}

%%% The glossary entry the acronym links to
%%%\newglossaryentry{mpog}{name={MPO},
%%%    description={Pêches et Océans Canada}}

%%% define the acronym and use the see= option
%%%\newglossaryentry{mpo}{type=\acronymtype, name={MPO}, description={Pêches et Océans Canada, anciennement le ministère des Pêches et des Océans}, first={Pêches et Océans Canada (MPO)\glsadd{mpog}}, see=[Glossary:]{mpog}}



\newglossaryentry{Ecosystem Survey} {
  name=Ecosystem Survey,
  description={
    The Ecosystem Survey app is used to as the main data input interface.
    }
}

%%% The glossary entry the acronym links to
\newglossaryentry{eseg}{name={ESE},
    description={The Ecosystem Survey Entry replaced the \gls{gse} in the early 2000s}}

%%% define the acronym and use the see= option
\newglossaryentry{ese}{type=\acronymtype, name={ESE}, description={Ecosystem Survey Entry}, first={Ecosystem Survey Entry (ESE)\glsadd{eseg}}, see=[Glossary:]{eseg}}


\newglossaryentry{Fishing Event} {
  name=Fishing Event,
  description={
    An event related to fishing activty of \gls{Set}.
    Metadata such as GPS coordinates and time to are associated with the folllowing event types:
    \emph{net deployed},
    \emph{doors deployed},
    \emph{warp deployed},
    \emph{net on bottom},
    \emph{haul back},
    \emph{net off bottom},
    \emph{door recovered},
    \emph{net recovered},
    and \emph{aborted}.
    A \gls{Set} may contain a list of such event types.
    The Fishing Events are added to the \gls{Set} by a crewmember using the \gls{Bridge} application.
  }
}

%%% The glossary entry the acronym links to
\newglossaryentry{gseg}{name={GSE},
    description={The GSE is a data entry tool developed in the 1980s}}

%%% define the acronym and use the see= option
\newglossaryentry{gse}{type=\acronymtype, name={GSE}, description={Groundfish Survey Entry}, first={Groundfish Survey Entry (GSE)\glsadd{gseg}}, see=[Glossary:]{gseg}}


%%% The glossary entry the acronym links to
\newglossaryentry{imtsg}{name={IMTS},
    description={Information Management and Technology Services is the IT branch of DFO}}

%%% define the acronym and use the see= option
\newglossaryentry{imts}{type=\acronymtype, name={IMTS}, description={Information Management and Technology Services}, first={Information Management and Technology Services (IMTS)\glsadd{imtsg}}, see=[Glossary:]{imtsg}}


%%% The glossary entry the acronym links to
\newglossaryentry{lang}{name={LAN},
    description={A series of computers connected to each other and capable of communicating with each other over wired or wireless connections.}}

%%% define the acronym and use the see= option
\newglossaryentry{lan}{type=\acronymtype, name={LAN}, description={A Local Area Network}, first={local area network (LAN)\glsadd{lang}}, see=[Glossary:]{lang}}


\newglossaryentry{Mission} {
  name=Mission,
  description={
    Same as \gls{Cruise}
  }
}

\newglossaryentry{Open Set} {
  name=Open Set,
  description={
    A \gls{Set} that has not been closed. All new Sets will start in an open state and will remain so untill closed.
    Sets that are activated will automatically be opened.
    The \gls{Current Set} is always open
  },
  see={Set}
}

\newglossaryentry{ORM} {
  % type=\acronymtype,
  name=ORM,
  description={Object Relational Mapping are an abstraction of relational entities (database tables) as objects},
  text={ORM},
  first={Object Relational Model (ORM)}
}

\newglossaryentry{Port Sampling} {
  name=Port Sampling,
  description={Scientific program collecting samples from commercial fishing activities}
}

\newglossaryentry{Protocol} {
  name=Protocol,
  description={TODO}
}

\newglossaryentry{REST-API} {
  name=REST-API,
  description={TODO, just an API over the web yo}
}

\newglossaryentry{Sampling Protocol} {
  name=Sampling Protocol,
  description={The detailed dexcription of what data is to be collected during a scientfic cruise}
}

\newglossaryentry{Sampling Req} {
  name=Sampling Requirements,
  description={The species-specific requirements for samples to be collected during a scientific cruise}
}

\newglossaryentry{Set} {
  name=Set,
  description={
    A Set contains all fishing activity and sampling results for a particular \gls{Station}
  }
}

\newglossaryentry{Set Flag} {
  name=Set Flag,
  description={A flag used to indicate that the data contained within the set has failed to pass a specific validation tests.
  Closing an \gls{Open Set} with active flags can be done by overriding the validation mechanism},
  see=\gls{Set}
  }

\newglossaryentry{Station} {
  name=Station,
  description={A target location specificied by coordinates where a scientific activity is to take place}
}

\newglossaryentry{Species} {
  name=Species,
  description={An identifiable taxon that can be assigned to a species code}
}

\newglossaryentry{vb} {
  % type=\acronymtype,
  name=VB,
  description={Pre-.NET Visual Basic for Applications is the early version of Microsoft VBA and is no longer supported or updated by Microsoft},
  text={VB},
  first={Pre-.NET Visual Basic (VB)}
}

\newglossaryentry{vba} {
  % type=\acronymtype,
  name=VBA,
  description={Visual Basic for Applications is a programming language built into most desktop Microsoft Office applications. More details can be found on the Wikipedia page for this programming language and from Microsoft},
  text={VBA},
  first={Visual Basic for Applications (VBA)}
}

%%% The glossary entry the acronym links to
\newglossaryentry{vcsg} {name={VCS},
description={A Version Control System records changes to a file or set of files over time so that specific versions can be recalled later. For example, git is a VSC. }}

%%% define the acronym and use the see= option
\newglossaryentry{vcs} {
  type=\acronymtype,
  name={VCS},
  description={Version Control System},
  text={VCS},
  first={Version Control System (VCS)}
}

\newglossaryentry{mrr} {
  % type=\acronymtype,
  name=MRR,
  description={Module des Relevés de Recherche},
  text={MRR},
  first={"Module des Relevés de Recherche" (MRR)}
}


%%% The glossary entry the acronym links to
\newglossaryentry{andesg}{name={Andes},
    description={Another data entry system is an application developed by Fisheries and Oceans Canada to support data collection for a variety of scientific programs.}}

%%% define the acronym and use the see= option
\newglossaryentry{andes}{type=\acronymtype, name={Andes}, description={Another data entry system}, first={\emph{An}other \emph{d}ata \emph{e}ntry \emph{s}ystem (Andes)\glsadd{andesg}}, see=[Glossary:]{andesg}}


%%% The glossary entry the acronym links to
\newglossaryentry{ERDg}{name={ERD},
    description={An entity relationship diagram.}}

%%% define the acronym and use the see= option
\newglossaryentry{ERD}{type=\acronymtype, name={ERD}, description={Entity Relationship Diagram}, first={Entity Relationship Diagram\glsadd{ERDg}}, see=[Glossary:]{ERDg}}


%%% The glossary entry the acronym links to
\newglossaryentry{CSSg}{name={CSS},
    description={A Cascading Style Sheet .}}

%%% define the acronym and use the see= option
\newglossaryentry{CSS}{type=\acronymtype, name={CSS}, description={Cascading Style Sheet}, first={Cascading Style Sheet\glsadd{CSS}}, see=[Glossary:]{CSSg}}

\newglossaryentry{javascript} {
  name={javascript},
  description={TO DO}
}


%%% The glossary entry the acronym links to
\newglossaryentry{html5g}{name={html5},
    description={The Hypertext Markup Language is TO DO.}}

%%% define the acronym and use the see= option
\newglossaryentry{html5}{type=\acronymtype, name={html5}, description={Hypertext Markup Language}, first={HTML5\glsadd{html5}}, see=[Glossary:]{html5g}}


\glsaddall
\glsaddallunused
